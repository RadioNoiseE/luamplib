% \iffalse meta-comment -- by the way, this file contains UTF-8
%
% Copyright (C) 2008-2024 by Hans Hagen, Taco Hoekwater, Elie Roux,
% Manuel Pégourié-Gonnard, Philipp Gesang and Kim Dohyun.
% Currently maintained by the LuaLaTeX development team.
% Support: <lualatex-dev@tug.org>
%
% This work is under the GPL v2.0 license.
%
% This work consists of the main source file luamplib.dtx
% and the derived files
%    luamplib.sty, luamplib.lua and luamplib.pdf.
%
% Unpacking:
%    tex luamplib.dtx
%
% Documentation:
%    lualatex luamplib.dtx
%
%<*ignore>
\begingroup
  \def\x{LaTeX2e}%
\expandafter\endgroup
\ifcase 0\ifx\install y1\fi\expandafter
         \ifx\csname processbatchFile\endcsname\relax\else1\fi
         \ifx\fmtname\x\else 1\fi\relax
\else\csname fi\endcsname
%</ignore>
%<*install>
\input docstrip.tex
\Msg{************************************************************************}
\Msg{* Installation}
\Msg{* Package: luamplib - metapost package for LuaTeX.}
\Msg{************************************************************************}

\keepsilent
\askforoverwritefalse

\let\MetaPrefix\relax

\preamble

See source file '\inFileName' for licencing and contact information.

\endpreamble

\let\MetaPrefix\DoubleperCent

\generate{%
  \usedir{tex/luatex/luamplib}%
  \file{luamplib.sty}{\from{luamplib.dtx}{package}}%
}

\def\MetaPrefix{-- }

\def\luapostamble{%
  \MetaPrefix^^J%
  \MetaPrefix\space End of File `\outFileName'.%
}

\def\currentpostamble{\luapostamble}%

\generate{%
  \usedir{tex/luatex/luamplib}%
  \file{luamplib.lua}{\from{luamplib.dtx}{lua}}%
}

\obeyspaces
\Msg{************************************************************************}
\Msg{*}
\Msg{* To finish the installation you have to move the following}
\Msg{* files into a directory searched by TeX:}
\Msg{*}
\Msg{*     luamplib.sty luamplib.lua}
\Msg{*}
\Msg{* Happy TeXing!}
\Msg{*}
\Msg{************************************************************************}

\endbatchfile
%</install>
%<*ignore>
\fi
%</ignore>
%<*driver>
\NeedsTeXFormat{LaTeX2e}
\ProvidesFile{luamplib.drv}%
  [2024/01/25 v2.25.3 Interface for using the mplib library]%
\documentclass{ltxdoc}
\usepackage{metalogo,multicol,mdwlist,fancyvrb,xspace}
\usepackage[x11names]{xcolor}
%
\def\primarycolor{DodgerBlue4}  %%-> rgb  16  78 139 | #104e8b
\def\secondarycolor{Goldenrod4} %%-> rgb 139 105 200 | #8b6914
%
\usepackage[
    bookmarks=true,
   colorlinks=true,
    linkcolor=\primarycolor,
     urlcolor=\secondarycolor,
    citecolor=\primarycolor,
     pdftitle={The luamplib package},
   pdfsubject={Interface for using the mplib library},
    pdfauthor={Hans Hagen, Taco Hoekwater, Elie Roux, Philipp Gesang \& Kim Dohyun},
  pdfkeywords={luatex, lualatex, mplib, metapost}
]{hyperref}
\usepackage{fontspec}
\setmainfont[
  Numbers     = OldStyle,
  Ligatures   = TeX,
  BoldFont    = {Linux Libertine O Bold},
  ItalicFont  = {Linux Libertine O Italic},
  SlantedFont = {Linux Libertine O Italic},
]{Linux Libertine O}
\setmonofont[Ligatures=TeX,Scale=MatchLowercase]{InconsolataN}
%setsansfont[Ligatures=TeX]{Linux Biolinum O}
\setsansfont[Ligatures=TeX,Scale=MatchLowercase]{Iwona Medium}
%setmathfont{XITS Math}

\usepackage{hologo}

\newcommand\ConTeXt {Con\TeX t\xspace}

\newcommand*\email [1] {<\href{mailto:#1}{#1}>}
\newcommand \file       {\nolinkurl}
\newcommand \pk         {\textsf}

\begin{document}
  \DocInput{luamplib.dtx}%
\end{document}
%</driver>
% \fi
%
% \CheckSum{0}
%
% \CharacterTable
%  {Upper-case    \A\B\C\D\E\F\G\H\I\J\K\L\M\N\O\P\Q\R\S\T\U\V\W\X\Y\Z
%   Lower-case    \a\b\c\d\e\f\g\h\i\j\k\l\m\n\o\p\q\r\s\t\u\v\w\x\y\z
%   Digits        \0\1\2\3\4\5\6\7\8\9
%   Exclamation   \!     Double quote  \"     Hash (number) \#
%   Dollar        \$     Percent       \%     Ampersand     \&
%   Acute accent  \'     Left paren    \(     Right paren   \)
%   Asterisk      \*     Plus          \+     Comma         \,
%   Minus         \-     Point         \.     Solidus       \/
%   Colon         \:     Semicolon     \;     Less than     \<
%   Equals        \=     Greater than  \>     Question mark \?
%   Commercial at \@     Left bracket  \[     Backslash     \\
%   Right bracket \]     Circumflex    \^     Underscore    \_
%   Grave accent  \`     Left brace    \{     Vertical bar  \|
%   Right brace   \}     Tilde         \~}
%
% \title{The \textsf{luamplib} package}
% \author{Hans Hagen, Taco Hoekwater, Elie Roux, Philipp Gesang and Kim Dohyun\\
% Maintainer: LuaLaTeX Maintainers ---
% Support: \email{lualatex-dev@tug.org}}
% \date{2024/01/25 v2.25.3}
%
% \maketitle
%
% \begin{abstract}
% Package to have metapost code typeset directly in a document with \LuaTeX.
% \end{abstract}
%
% \section{Documentation}
%
% This packages aims at providing a simple way to typeset directly metapost
% code in a document with \LuaTeX. \LuaTeX\ is built with the lua
% \texttt{mplib} library, that runs metapost code. This package is basically a
% wrapper (in Lua) for the Lua \texttt{mplib} functions and some \TeX\
% functions to have the output of the \texttt{mplib} functions in the pdf.
%
% In the past,
% the package required PDF mode in order to output something.
% Starting with version 2.7 it works in DVI mode as well, though
% DVIPDFMx is the only DVI tool currently supported.
%
% The metapost figures are put in a \TeX\ \texttt{hbox} with dimensions
% adjusted to the metapost code.
%
% Using this package is easy: in Plain, type your metapost code between the
% macros \cs{mplibcode} and \cs{endmplibcode}, and in \LaTeX\ in the
% \texttt{mplibcode} environment.
%
% The code is from the \texttt{luatex-mplib.lua} and \texttt{luatex-mplib.tex} files
% from \ConTeXt, they have been adapted to \LaTeX\ and Plain by Elie Roux and
% Philipp Gesang, new functionalities have been added by Kim Dohyun.
% The changes are:
%
% \begin{itemize}
% \item a \LaTeX\ environment
% \item all \TeX\ macros start by |mplib|
% \item use of luatexbase for errors, warnings and declaration
% \item possibility to use |btex ... etex| to typeset \TeX\ code.
%   |textext()| is a more versatile macro equivalent to |TEX()| from TEX.mp.
%   |TEX()| is also allowed and is a synomym of |textext()|.\par\smallskip
%   \textsc{n.b.} Since v2.5, |btex ... etex| input from external |mp| files
%   will also be processed by \textsf{luamplib}.\par\smallskip
%   \textsc{n.b.} Since v2.20, |verbatimtex ... etex| from external |mp| files
%    will be also processed by \textsf{luamplib}. Warning: This is a change
%   from previous version.
% \end{itemize}
%
% Some more changes and cautions are:
%
% \paragraph{\cs{mplibforcehmode}}
%   When this macro is declared, every mplibcode figure box will be
%   typeset in horizontal mode, so \cs{centering}, \cs{raggedleft} etc
%   will have effects. |\mplibnoforcehmode|, being default, reverts this
%   setting. (Actually these commands redefine |\prependtomplibbox|. You
%   can define this command with anything suitable before a box.)
%
% \paragraph{\cs{mpliblegacybehavior\{enable\}}}
%   By default, |\mpliblegacybehavior{enable}| is already declared,
%   in which case
%   a |verbatimtex ... etex| that comes just before |beginfig()|
%   is not ignored, but the \TeX\ code will be inserted before the
%   following mplib hbox.  Using this command,
%   each mplib box can be freely moved horizontally and/or vertically.
%   Also, a box number might be assigned to mplib box, allowing it to be
%   reused later (see test files).
%   \begin{verbatim}
%     \mplibcode
%     verbatimtex \moveright 3cm etex; beginfig(0); ... endfig;
%     verbatimtex \leavevmode etex; beginfig(1); ... endfig;
%     verbatimtex \leavevmode\lower 1ex etex; beginfig(2); ... endfig;
%     verbatimtex \endgraf\moveright 1cm etex; beginfig(3); ... endfig;
%     \endmplibcode
%   \end{verbatim}
%   \textsc{n.b.} \cs{endgraf} should be used instead of \cs{par} inside
%   |verbatimtex ... etex|.
%
%   By contrast, \TeX\ code in |VerbatimTeX(...)| or |verbatimtex ... etex|
%   between |beginfig()| and |endfig| will be inserted
%   after flushing out the mplib figure.
%   \begin{verbatim}
%     \mplibcode
%       D := sqrt(2)**7;
%       beginfig(0);
%       draw fullcircle scaled D;
%       VerbatimTeX("\gdef\Dia{" & decimal D & "}");
%       endfig;
%     \endmplibcode
%     diameter: \Dia bp.
%   \end{verbatim}
%
% \paragraph{\cs{mpliblegacybehavior\{disable\}}}
%   If |\mpliblegacybehavior{disabled}| is declared by user, any
%   |verbatimtex ... etex| will be executed, along with |btex ... etex|,
%   sequentially one by one.
%   So, some \TeX\ code in |verbatimtex ... etex| will have effects on
%   |btex ... etex| codes that follows.
%   \begin{verbatim}
%     \begin{mplibcode}
%       beginfig(0);
%       draw btex ABC etex;
%       verbatimtex \bfseries etex;
%       draw btex DEF etex shifted (1cm,0); % bold face
%       draw btex GHI etex shifted (2cm,0); % bold face
%       endfig;
%     \end{mplibcode}
%   \end{verbatim}
%
% \paragraph{About figure box metrics}
%   Notice that, after each figure is processed, macro \cs{MPwidth} stores
%   the width value of latest figure; \cs{MPheight}, the height value.
%   Incidentally, also note that \cs{MPllx}, \cs{MPlly}, \cs{MPurx}, and
%   \cs{MPury} store the bounding box information of latest figure
%   without the unit |bp|.
%
% \paragraph{\cs{everymplib}, \cs{everyendmplib}}
%   Since v2.3, new macros \cs{everymplib} and \cs{everyendmplib} redefine
%   the lua table containing MetaPost code
%   which will
%   be automatically inserted at the beginning and ending of each |mplibcode|.
%   \begin{verbatim}
%     \everymplib{ beginfig(0); }
%     \everyendmplib{ endfig; }
%     \mplibcode % beginfig/endfig not needed
%       draw fullcircle scaled 1cm;
%     \endmplibcode
%   \end{verbatim}
%
% \paragraph{\cs{mpdim}}
%   Since v2.3, \cs{mpdim} and other raw \TeX\ commands are allowed
%   inside mplib code. This feature is inpired by gmp.sty authored by
%   Enrico Gregorio. Please refer the manual of gmp package for details.
%   \begin{verbatim}
%     \begin{mplibcode}
%       draw origin--(\mpdim{\linewidth},0) withpen pencircle scaled 4
%       dashed evenly scaled 4 withcolor \mpcolor{orange};
%     \end{mplibcode}
%   \end{verbatim}
%   \textsc{n.b.} Users should not use the protected variant of
%   |btex ... etex| as provided by gmp package. As \textsf{luamplib}
%   automatically protects \TeX\ code inbetween, \cs{btex} is not supported
%   here.
%
% \paragraph{\cs{mpcolor}}
%   With \cs{mpcolor} command, color names or expressions of
%   \textsf{color}/\textsf{xcolor} packages can be used inside mplibcode
%   enviroment (after |withcolor| operator),
%   though \textsf{luamplib} does not automatically load these
%   packages. See the example code above. For spot colors, \textsf{(x)spotcolor}
%   (in PDF mode) and \textsf{xespotcolor} (in DVI mode) packages are supported
%   as well.
%
% \paragraph{\cs{mplibnumbersystem}}
%   Users can choose |numbersystem| option since v2.4.
%   The default value |scaled| can be changed to |double| or |decimal|
%   by declaring |\mplibnumbersystem{double}| or |\mplibnumbersystem{decimal}|.
%   For details see
%   \url{http://github.com/lualatex/luamplib/issues/21}.
%
% \paragraph{Settings regarding cache files}
%   To support |btex ... etex| in external |.mp| files, \textsf{luamplib}
%   inspects the content of each and every |.mp| input files and makes caches
%   if nececcsary, before returning their paths to \LuaTeX's mplib library.
%   This would make the compilation time longer wastefully, as most |.mp| files
%   do not contain |btex ... etex| command.  So \textsf{luamplib} provides
%   macros as follows, so that users can give instruction about files
%   that do not require this functionality.
%   \begin{itemize}
%   \item |\mplibmakenocache{<filename>[,<filename>,...]}|
%   \item |\mplibcancelnocache{<filename>[,<filename>,...]}|
%   \end{itemize}
%   where |<filename>| is a file name excluding |.mp| extension.
%   Note that |.mp| files under |$TEXMFMAIN/metapost/base| and
%   |$TEXMFMAIN/metapost/context/base| are already registered by default.
%
%   By default, cache files will be stored in |$TEXMFVAR/luamplib_cache| or,
%   if it's not available, in the same directory as where pdf/dvi output file
%   is saved. This however can be changed by the command
%   |\mplibcachedir{<directory path>}|, where tilde (|~|) is interpreted
%   as the user's home directory (on a windows machine as well).
%   As backslashes (|\|) should be escaped by users, it would be easier to use
%   slashes (|/|) instead.
%
% \paragraph{\cs{mplibtextextlabel}}
%   Starting with v2.6, |\mplibtextextlabel{enable}| enables
%   string labels typeset via |textext()| instead of |infont| operator.
%   So, |label("my text",origin)| thereafter is exactly the same as
%   |label(textext("my text"),origin)|. \textsc{n.b.} In the background,
%   \textsf{luamplib} redefines |infont| operator so that the right side
%   argument (the font part) is totally ignored. Every string label
%   therefore will be typeset with current \TeX\ font.
%   Also take care of |char| operator in the left side argument,
%   as this might bring unpermitted characters into \TeX.
%
% \paragraph{\cs{mplibcodeinherit}}
%   Starting with v2.9, |\mplibcodeinherit{enable}| enables the inheritance
%   of variables, constants, and macros defined by previous |mplibcode| chunks.
%   On the contrary, the default value |\mplibcodeinherit{disable}| will make
%   each code chunks being treated as an independent instance, and never
%   affected by previous code chunks.
%
% \paragraph{Separate instances for \LaTeX{} environment}
%   v2.22 has added the support for several named MetaPost instances
%   in \LaTeX{} |mplibcode| environment.
%   Syntax is like so:
%   \begin{verbatim}
%   \begin{mplibcode}[instanceName]
%     % some mp code
%   \end{mplibcode}
%   \end{verbatim}
%   Behaviour is as follows.
%   \begin{itemize}
%   \item  All the variables and functions are shared
%     only among all the environments belonging to the same instance.
%   \item  |\mplibcodeinherit| only affects environments
%     with no instance name set (since if a name is set,
%     the code is intended to be reused at some point).
%   \item  |btex ... etex| labels still exist separately and
%     require |\mplibglobaltextext|.
%   \item  When an instance names is set,
%     respective |\currentmpinstancename| is set.
%   \end{itemize}
%   In parellel with this functionality, v2.23 and after supports
%   optional argument of instance name for \cs{everymplib} and
%   \cs{everyendmplib}, affecting only those |mplibcode| environments
%   of the same name.
%   Unnamed \cs{everymplib} affects not only those instances with no name,
%   but also those with name but with no corresponding \cs{everymplib}.
%   Syntax is:
%   \begin{verbatim}
%     \everymplib[instanceName]{...}
%     \everyendmplib[instanceName]{...}
%   \end{verbatim}
%
% \paragraph{\cs{mplibglobaltextext}}
%   To inherit |btex ... etex| labels as well as metapost variables,
%   it is necessary to declare \cs{mplibglobaltextext\{enable\}} in advance.
%   On this case, be careful that normal \TeX\ boxes can conflict with
%   |btex ... etex| boxes, though this would occur very rarely.
%   Notwithstanding the danger, it is a `must' option to activate
%   \cs{mplibglobaltextext} if you want to use |graph.mp|
%   with \cs{mplibcodeinherit} functionality.
%   \begin{verbatim}
%   \mplibcodeinherit{enable}
%   \mplibglobaltextext{enable}
%   \everymplib{ beginfig(0);} \everyendmplib{ endfig;}
%   \mplibcode
%     label(btex $\sqrt{2}$ etex, origin);
%     draw fullcircle scaled 20;
%     picture pic; pic := currentpicture;
%   \endmplibcode
%   \mplibcode
%     currentpicture := pic scaled 2;
%   \endmplibcode
%   \end{verbatim}
%
% \paragraph{\cs{mplibverbatim}}
%   Starting with v2.11, users can issue |\mplibverbatim{enable}|, after which
%   the contents of mplibcode environment will be read verbatim. As a result,
%   except for |\mpdim| and |\mpcolor|, all other \TeX\ commands outside
%   |btex ... etex| or |verbatimtex ... etex| are not expanded and will be fed
%   literally into the mplib process.
%
% \paragraph{\cs{mplibshowlog}}
%   When |\mplibshowlog{enable}| is declared, log messages returned by
%   |mplib| instance will be printed into the |.log| file.
%   |\mplibshowlog{disable}| will revert this functionality.
%   This is a \TeX{} side interface for |luamplib.showlog|. (v2.20.8)
%
% \paragraph{luamplib.cfg}
%   At the end of package loading, \textsf{luamplib} searches
%   |luamplib.cfg| and, if found, reads the file in automatically.
%   Frequently used settings such as \cs{everymplib} or \cs{mplibforcehmode}
%   are suitable for going into this file.
%
% \bigskip
%
% There are (basically) two formats for metapost: \emph{plain} and
% \emph{metafun}. By default, the \emph{plain} format is used, but you can set
% the format to be used by future figures at any time using
% \cs{mplibsetformat}\marg{format name}.
%
%    \section{Implementation}
%
%    \subsection{Lua module}
%
% \iffalse
%<*lua>
% \fi
%
%    \begin{macrocode}

luatexbase.provides_module {
  name          = "luamplib",
  version       = "2.25.3",
  date          = "2024/01/25",
  description   = "Lua package to typeset Metapost with LuaTeX's MPLib.",
}

local format, abs = string.format, math.abs

local err  = function(...)
  return luatexbase.module_error  ("luamplib", select("#",...) > 1 and format(...) or ...)
end
local warn = function(...)
  return luatexbase.module_warning("luamplib", select("#",...) > 1 and format(...) or ...)
end
local info = function(...)
  return luatexbase.module_info   ("luamplib", select("#",...) > 1 and format(...) or ...)
end

%    \end{macrocode}
%
%    Use the |luamplib| namespace, since |mplib| is for the metapost library
%    itself. \ConTeXt{} uses |metapost|.
%    \begin{macrocode}
luamplib          = luamplib or { }
local luamplib    = luamplib

luamplib.showlog  = luamplib.showlog or false

%    \end{macrocode}
%
%    This module is a stripped down version of libraries that are used by
%    \ConTeXt. Provide a few ``shortcuts'' expected by the imported code.
%    \begin{macrocode}
local tableconcat   = table.concat
local texsprint     = tex.sprint
local textprint     = tex.tprint

local texget      = tex.get
local texgettoks  = tex.gettoks
local texgetbox   = tex.getbox
local texruntoks  = tex.runtoks
%    \end{macrocode}
%
%    We don't use |tex.scantoks| anymore. See below reagrding |tex.runtoks|.
%    \begin{verbatim}
%      local texscantoks = tex.scantoks
%    \end{verbatim}
%    \begin{macrocode}

if not texruntoks then
  err("Your LuaTeX version is too old. Please upgrade it to the latest")
end

local mplib = require ('mplib')
local kpse  = require ('kpse')
local lfs   = require ('lfs')

local lfsattributes = lfs.attributes
local lfsisdir      = lfs.isdir
local lfsmkdir      = lfs.mkdir
local lfstouch      = lfs.touch
local ioopen        = io.open

%    \end{macrocode}
%
%    Some helper functions, prepared for the case when |l-file| etc
%    is not loaded.
%    \begin{macrocode}
local file = file or { }
local replacesuffix = file.replacesuffix or function(filename, suffix)
  return (filename:gsub("%.[%a%d]+$","")) .. "." .. suffix
end

local is_writable = file.is_writable or function(name)
  if lfsisdir(name) then
    name = name .. "/_luam_plib_temp_file_"
    local fh = ioopen(name,"w")
    if fh then
      fh:close(); os.remove(name)
      return true
    end
  end
end
local mk_full_path = lfs.mkdirs or function(path)
  local full = ""
  for sub in path:gmatch("(/*[^\\/]+)") do
    full = full .. sub
    lfsmkdir(full)
  end
end

%    \end{macrocode}
%
%    |btex ... etex| in input |.mp| files will be replaced in finder.
%    Because of the limitation of MPLib regarding |make_text|,
%    we might have to make cache files modified from input files.
%    \begin{macrocode}
local luamplibtime = kpse.find_file("luamplib.lua")
luamplibtime = luamplibtime and lfsattributes(luamplibtime,"modification")

local currenttime = os.time()

local outputdir
if lfstouch then
  local texmfvar = kpse.var_value'TEXMFVAR'
  if texmfvar and texmfvar ~= "" then
    for _,dir in next, texmfvar:explode(os.type == "windows" and ";" or ":") do
      if not lfsisdir(dir) then
        mk_full_path(dir)
      end
      if is_writable(dir) then
        local cached = format("%s/luamplib_cache",dir)
        lfsmkdir(cached)
        outputdir = cached
        break
      end
    end
  end
end
if not outputdir then
  outputdir = kpse.var_value'TEXMF_OUTPUT_DIRECTORY' or "."
end

function luamplib.getcachedir(dir)
  dir = dir:gsub("##","#")
  dir = dir:gsub("^~",
    os.type == "windows" and os.getenv("UserProfile") or os.getenv("HOME"))
  if lfstouch and dir then
    if lfsisdir(dir) then
      if is_writable(dir) then
        luamplib.cachedir = dir
      else
        warn("Directory '%s' is not writable!", dir)
      end
    else
      warn("Directory '%s' does not exist!", dir)
    end
  end
end

%    \end{macrocode}
%
%    Some basic MetaPost files not necessary to make cache files.
%    \begin{macrocode}
local noneedtoreplace = {
  ["boxes.mp"] = true, --  ["format.mp"] = true,
  ["graph.mp"] = true, ["marith.mp"] = true, ["mfplain.mp"] = true,
  ["mpost.mp"] = true, ["plain.mp"] = true, ["rboxes.mp"] = true,
  ["sarith.mp"] = true, ["string.mp"] = true, -- ["TEX.mp"] = true,
  ["metafun.mp"] = true, ["metafun.mpiv"] = true, ["mp-abck.mpiv"] = true,
  ["mp-apos.mpiv"] = true, ["mp-asnc.mpiv"] = true, ["mp-bare.mpiv"] = true,
  ["mp-base.mpiv"] = true, ["mp-blob.mpiv"] = true, ["mp-butt.mpiv"] = true,
  ["mp-char.mpiv"] = true, ["mp-chem.mpiv"] = true, ["mp-core.mpiv"] = true,
  ["mp-crop.mpiv"] = true, ["mp-figs.mpiv"] = true, ["mp-form.mpiv"] = true,
  ["mp-func.mpiv"] = true, ["mp-grap.mpiv"] = true, ["mp-grid.mpiv"] = true,
  ["mp-grph.mpiv"] = true, ["mp-idea.mpiv"] = true, ["mp-luas.mpiv"] = true,
  ["mp-mlib.mpiv"] = true, ["mp-node.mpiv"] = true, ["mp-page.mpiv"] = true,
  ["mp-shap.mpiv"] = true, ["mp-step.mpiv"] = true, ["mp-text.mpiv"] = true,
  ["mp-tool.mpiv"] = true, ["mp-cont.mpiv"] = true,
}
luamplib.noneedtoreplace = noneedtoreplace

%    \end{macrocode}
%
%    |format.mp| is much complicated, so specially treated.
%    \begin{macrocode}
local function replaceformatmp(file,newfile,ofmodify)
  local fh = ioopen(file,"r")
  if not fh then return file end
  local data = fh:read("*all"); fh:close()
  fh = ioopen(newfile,"w")
  if not fh then return file end
  fh:write(
    "let normalinfont = infont;\n",
    "primarydef str infont name = rawtextext(str) enddef;\n",
    data,
    "vardef Fmant_(expr x) = rawtextext(decimal abs x) enddef;\n",
    "vardef Fexp_(expr x) = rawtextext(\"$^{\"&decimal x&\"}$\") enddef;\n",
    "let infont = normalinfont;\n"
  ); fh:close()
  lfstouch(newfile,currenttime,ofmodify)
  return newfile
end

%    \end{macrocode}
%
%    Replace |btex ... etex| and |verbatimtex ... etex| in input files,
%    if needed.
%    \begin{macrocode}
local name_b = "%f[%a_]"
local name_e = "%f[^%a_]"
local btex_etex = name_b.."btex"..name_e.."%s*(.-)%s*"..name_b.."etex"..name_e
local verbatimtex_etex = name_b.."verbatimtex"..name_e.."%s*(.-)%s*"..name_b.."etex"..name_e

local function replaceinputmpfile (name,file)
  local ofmodify = lfsattributes(file,"modification")
  if not ofmodify then return file end
  local cachedir = luamplib.cachedir or outputdir
  local newfile = name:gsub("%W","_")
  newfile = cachedir .."/luamplib_input_"..newfile
  if newfile and luamplibtime then
    local nf = lfsattributes(newfile)
    if nf and nf.mode == "file" and
      ofmodify == nf.modification and luamplibtime < nf.access then
      return nf.size == 0 and file or newfile
    end
  end

  if name == "format.mp" then return replaceformatmp(file,newfile,ofmodify) end

  local fh = ioopen(file,"r")
  if not fh then return file end
  local data = fh:read("*all"); fh:close()

%    \end{macrocode}
%
%    ``|etex|'' must be followed by a space or semicolon as specified in
%    \LuaTeX\ manual, which is not the case of standalone MetaPost though.
%    \begin{macrocode}
  local count,cnt = 0,0
  data, cnt = data:gsub(btex_etex, "btex %1 etex ") -- space
  count = count + cnt
  data, cnt = data:gsub(verbatimtex_etex, "verbatimtex %1 etex;") -- semicolon
  count = count + cnt

  if count == 0 then
    noneedtoreplace[name] = true
    fh = ioopen(newfile,"w");
    if fh then
      fh:close()
      lfstouch(newfile,currenttime,ofmodify)
    end
    return file
  end

  fh = ioopen(newfile,"w")
  if not fh then return file end
  fh:write(data); fh:close()
  lfstouch(newfile,currenttime,ofmodify)
  return newfile
end

%    \end{macrocode}
%
%    As the finder function for MPLib, use the |kpse| library and
%    make it behave like as if MetaPost was used. And replace it with
%    cache files if needed.
%    See also \#74, \#97.
%    \begin{macrocode}
local mpkpse
do
  local exe = 0
  while arg[exe-1] do
    exe = exe-1
  end
  mpkpse = kpse.new(arg[exe], "mpost")
end

local special_ftype = {
  pfb = "type1 fonts",
  enc = "enc files",
}

local function finder(name, mode, ftype)
  if mode == "w" then
    if name and name ~= "mpout.log" then
      kpse.record_output_file(name) -- recorder
    end
    return name
  else
    ftype = special_ftype[ftype] or ftype
    local file = mpkpse:find_file(name,ftype)
    if file then
      if lfstouch and ftype == "mp" and not noneedtoreplace[name] then
        file = replaceinputmpfile(name,file)
      end
    else
      file = mpkpse:find_file(name, name:match("%a+$"))
    end
    if file then
      kpse.record_input_file(file) -- recorder
    end
    return file
  end
end
luamplib.finder = finder

%    \end{macrocode}
%
%    Create and load MPLib instances.
%    We do not support ancient version of MPLib any more.
%    (Don't know which version of MPLib started to support
%    |make_text| and |run_script|; let the users find it.)
%    \begin{macrocode}
if tonumber(mplib.version()) <= 1.50 then
  err("luamplib no longer supports mplib v1.50 or lower. "..
  "Please upgrade to the latest version of LuaTeX")
end

local preamble = [[
  boolean mplib ; mplib := true ;
  let dump = endinput ;
  let normalfontsize = fontsize;
  input %s ;
]]

local logatload
local function reporterror (result, indeed)
  if not result then
    err("no result object returned")
  else
    local t, e, l = result.term, result.error, result.log
%    \end{macrocode}
%
%    log has more information than term, so log first (2021/08/02)
%    \begin{macrocode}
    local log = l or t or "no-term"
    log = log:gsub("%(Please type a command or say `end'%)",""):gsub("\n+","\n")
    if result.status > 0 then
      warn(log)
      if result.status > 1 then
        err(e or "see above messages")
      end
    elseif indeed then
      local log = logatload..log
%    \end{macrocode}
%
%    v2.6.1: now luamplib does not disregard |show| command,
%    even when |luamplib.showlog| is false.  Incidentally,
%    it does not raise error but just prints a warning,
%    even if output has no figure.
%    \begin{macrocode}
      if log:find"\n>>" then
        warn(log)
      elseif log:find"%g" then
        if luamplib.showlog then
          info(log)
        elseif not result.fig then
          info(log)
        end
      end
      logatload = ""
    else
      logatload = log
    end
    return log
  end
end

local function luamplibload (name)
  local mpx = mplib.new {
    ini_version = true,
    find_file   = luamplib.finder,
%    \end{macrocode}
%
%    Make use of |make_text| and |run_script|, which will co-operate
%    with \LuaTeX's |tex.runtoks|. And we
%    provide |numbersystem| option since v2.4. Default value ``|scaled|''
%    can be changed by declaring |\mplibnumbersystem{double}|
%    or |\mplibnumbersystem{decimal}|.
%    See \url{https://github.com/lualatex/luamplib/issues/21}.
%    \begin{macrocode}
    make_text   = luamplib.maketext,
    run_script  = luamplib.runscript,
    math_mode   = luamplib.numbersystem,
    job_name    = tex.jobname,
    random_seed = math.random(4095),
    extensions  = 1,
  }
%    \end{macrocode}
%
%    Append our own MetaPost preamble to the preamble above.
%    \begin{macrocode}
  local preamble = preamble .. luamplib.mplibcodepreamble
  if luamplib.legacy_verbatimtex then
    preamble = preamble .. luamplib.legacyverbatimtexpreamble
  end
  if luamplib.textextlabel then
    preamble = preamble .. luamplib.textextlabelpreamble
  end
  local result
  if not mpx then
    result = { status = 99, error = "out of memory"}
  else
    result = mpx:execute(format(preamble, replacesuffix(name,"mp")))
  end
  reporterror(result)
  return mpx, result
end

%    \end{macrocode}
%
%    |plain| or |metafun|,
%    though we cannot support |metafun| format fully.
%    \begin{macrocode}
local currentformat = "plain"

local function setformat (name)
  currentformat = name
end
luamplib.setformat = setformat

%    \end{macrocode}
%
%    Here, excute each |mplibcode| data,
%    ie |\begin{mplibcode} ... \end{mplibcode}|.
%    \begin{macrocode}
local function process_indeed (mpx, data)
  local converted, result = false, {}
  if mpx and data then
    result = mpx:execute(data)
    local log = reporterror(result, true)
    if log then
      if result.fig then
        converted = luamplib.convert(result)
      else
        warn("No figure output. Maybe no beginfig/endfig")
      end
    end
  else
    err("Mem file unloadable. Maybe generated with a different version of mplib?")
  end
  return converted, result
end

%    \end{macrocode}
%
%    v2.9 has introduced the concept of ``code inherit''
%    \begin{macrocode}
luamplib.codeinherit = false
local mplibinstances = {}

local function process (data, instancename)
%    \end{macrocode}
%
%    The workaround of issue \#70 seems to be unnecessary, as we use
%    |make_text| now.
%    \begin{verbatim}
%    if not data:find(name_b.."beginfig%s*%([%+%-%s]*%d[%.%d%s]*%)") then
%      data = data .. "beginfig(-1);endfig;"
%    end
%    \end{verbatim}
%    \begin{macrocode}
  local defaultinstancename = currentformat .. (luamplib.numbersystem or "scaled")
    .. tostring(luamplib.textextlabel) .. tostring(luamplib.legacy_verbatimtex)
  local currfmt = instancename or defaultinstancename
  if #currfmt == 0 then
    currfmt = defaultinstancename
  end
  local mpx = mplibinstances[currfmt]
  local standalone = false
  if currfmt == defaultinstancename then
    standalone = not luamplib.codeinherit
  end
  if mpx and standalone then
    mpx:finish()
  end
  if standalone or not mpx then
    mpx = luamplibload(currentformat)
    mplibinstances[currfmt] = mpx
  end
  return process_indeed(mpx, data)
end

%    \end{macrocode}
%
%    |make_text| and some |run_script| uses \LuaTeX's |tex.runtoks|,
%    which made possible running \TeX\ code snippets inside |\directlua|.
%    \begin{macrocode}
local catlatex = luatexbase.registernumber("catcodetable@latex")
local catat11  = luatexbase.registernumber("catcodetable@atletter")

%    \end{macrocode}
%
%    |tex.scantoks| sometimes fail to read catcode properly, especially
%    |\#|, |\&|, or |\%|. After some experiment, we dropped using it.
%    Instead, a function containing |tex.script| seems to work nicely.
%    \begin{verbatim}
%      local function run_tex_code_no_use (str, cat)
%        cat = cat or catlatex
%        texscantoks("mplibtmptoks", cat, str)
%        texruntoks("mplibtmptoks")
%      end
%    \end{verbatim}
%    \begin{macrocode}
local function run_tex_code (str, cat)
  cat = cat or catlatex
  texruntoks(function() texsprint(cat, str) end)
end

%    \end{macrocode}
%
%    Indefinite number of boxes are needed for |btex ... etex|.
%    So starts at somewhat huge number of box registry. Of course,
%    this may conflict with other packages using many many boxes.
%    (When |codeinherit| feature is enabled, boxes must be globally defined.)
%    But I don't know any reliable way to escape this danger.
%    \begin{macrocode}
local tex_box_id = 2047
%    \end{macrocode}
%
%    For conversion of |sp| to |bp|.
%    \begin{macrocode}
local factor = 65536*(7227/7200)

local textext_fmt = [[image(addto currentpicture doublepath unitsquare ]]..
  [[xscaled %f yscaled %f shifted (0,-%f) ]]..
  [[withprescript "mplibtexboxid=%i:%f:%f")]]

local function process_tex_text (str)
  if str then
    tex_box_id = tex_box_id + 1
    local global = luamplib.globaltextext and "\\global" or ""
    run_tex_code(format("%s\\setbox%i\\hbox{%s}", global, tex_box_id, str))
    local box = texgetbox(tex_box_id)
    local wd  = box.width  / factor
    local ht  = box.height / factor
    local dp  = box.depth  / factor
    return textext_fmt:format(wd, ht+dp, dp, tex_box_id, wd, ht+dp)
  end
  return ""
end

%    \end{macrocode}
%
%    Make |color| or |xcolor|'s color expressions usable,
%    with \cs{mpcolor} or |mplibcolor|. These commands should be used
%    with graphical objects.
%    \begin{macrocode}
local mplibcolor_fmt = [[\begingroup\let\XC@mcolor\relax]]..
  [[\def\set@color{\global\mplibtmptoks\expandafter{\current@color}}]]..
  [[\color %s \endgroup]]

local function process_color (str)
  if str then
    if not str:find("{.-}") then
      str = format("{%s}",str)
    end
    run_tex_code(mplibcolor_fmt:format(str), catat11)
    return format('1 withprescript "MPlibOverrideColor=%s"', texgettoks"mplibtmptoks")
  end
  return ""
end

%    \end{macrocode}
%
%    \cs{mpdim} is expanded before MPLib process, so code below will not be
%    used for |mplibcode| data. But who knows anyone would want it
%    in |.mp| input file. If then, you can say |mplibdimen(".5\textwidth")|
%    for example.
%    \begin{macrocode}
local function process_dimen (str)
  if str then
    str = str:gsub("{(.+)}","%1")
    run_tex_code(format([[\mplibtmptoks\expandafter{\the\dimexpr %s\relax}]], str))
    return format("begingroup %s endgroup", texgettoks"mplibtmptoks")
  end
  return ""
end

%    \end{macrocode}
%
%    Newly introduced method of processing |verbatimtex ... etex|.
%    Used when |\mpliblegacybehavior{false}| is declared.
%    \begin{macrocode}
local function process_verbatimtex_text (str)
  if str then
    run_tex_code(str)
  end
  return ""
end

%    \end{macrocode}
%
%    For legacy verbatimtex process.
%    |verbatimtex ... etex| before |beginfig()| is not ignored,
%    but the \TeX\ code is inserted just before the mplib box. And
%    \TeX\ code inside |beginfig() ... endfig| is inserted after the mplib box.
%    \begin{macrocode}
local tex_code_pre_mplib = {}
luamplib.figid = 1
luamplib.in_the_fig = false

local function legacy_mplibcode_reset ()
  tex_code_pre_mplib = {}
  luamplib.figid = 1
end

local function process_verbatimtex_prefig (str)
  if str then
    tex_code_pre_mplib[luamplib.figid] = str
  end
  return ""
end

local function process_verbatimtex_infig (str)
  if str then
    return format('special "postmplibverbtex=%s";', str)
  end
  return ""
end

local runscript_funcs = {
  luamplibtext    = process_tex_text,
  luamplibcolor   = process_color,
  luamplibdimen   = process_dimen,
  luamplibprefig  = process_verbatimtex_prefig,
  luamplibinfig   = process_verbatimtex_infig,
  luamplibverbtex = process_verbatimtex_text,
}

%    \end{macrocode}
%
%    For |metafun| format. see issue \#79.
%    \begin{macrocode}
mp = mp or {}
local mp = mp
mp.mf_path_reset = mp.mf_path_reset or function() end
mp.mf_finish_saving_data = mp.mf_finish_saving_data or function() end

%    \end{macrocode}
%
%    metafun 2021-03-09 changes crashes luamplib.
%    \begin{macrocode}
catcodes = catcodes or {}
local catcodes = catcodes
catcodes.numbers = catcodes.numbers or {}
catcodes.numbers.ctxcatcodes = catcodes.numbers.ctxcatcodes or catlatex
catcodes.numbers.texcatcodes = catcodes.numbers.texcatcodes or catlatex
catcodes.numbers.luacatcodes = catcodes.numbers.luacatcodes or catlatex
catcodes.numbers.notcatcodes = catcodes.numbers.notcatcodes or catlatex
catcodes.numbers.vrbcatcodes = catcodes.numbers.vrbcatcodes or catlatex
catcodes.numbers.prtcatcodes = catcodes.numbers.prtcatcodes or catlatex
catcodes.numbers.txtcatcodes = catcodes.numbers.txtcatcodes or catlatex

%    \end{macrocode}
%
%    A function from \ConTeXt\ general.
%    \begin{macrocode}
local function mpprint(buffer,...)
  for i=1,select("#",...) do
    local value = select(i,...)
    if value ~= nil then
      local t = type(value)
      if t == "number" then
        buffer[#buffer+1] = format("%.16f",value)
      elseif t == "string" then
        buffer[#buffer+1] = value
      elseif t == "table" then
        buffer[#buffer+1] = "(" .. tableconcat(value,",") .. ")"
      else -- boolean or whatever
        buffer[#buffer+1] = tostring(value)
      end
    end
  end
end

function luamplib.runscript (code)
  local id, str = code:match("(.-){(.*)}")
  if id and str then
    local f = runscript_funcs[id]
    if f then
      local t = f(str)
      if t then return t end
    end
  end
  local f = loadstring(code)
  if type(f) == "function" then
    local buffer = {}
    function mp.print(...)
      mpprint(buffer,...)
    end
    f()
    buffer = tableconcat(buffer)
    if buffer and buffer ~= "" then
      return buffer
    end
    buffer = {}
    mpprint(buffer, f())
    return tableconcat(buffer)
  end
  return ""
end

%    \end{macrocode}
%
%    |make_text| must be one liner, so comment sign is not allowed.
%    \begin{macrocode}
local function protecttexcontents (str)
  return str:gsub("\\%%", "\0PerCent\0")
            :gsub("%%.-\n", "")
            :gsub("%%.-$",  "")
            :gsub("%zPerCent%z", "\\%%")
            :gsub("%s+", " ")
end

luamplib.legacy_verbatimtex = true

function luamplib.maketext (str, what)
  if str and str ~= "" then
    str = protecttexcontents(str)
    if what == 1 then
      if not str:find("\\documentclass"..name_e) and
         not str:find("\\begin%s*{document}") and
         not str:find("\\documentstyle"..name_e) and
         not str:find("\\usepackage"..name_e) then
        if luamplib.legacy_verbatimtex then
          if luamplib.in_the_fig then
            return process_verbatimtex_infig(str)
          else
            return process_verbatimtex_prefig(str)
          end
        else
          return process_verbatimtex_text(str)
        end
      end
    else
      return process_tex_text(str)
    end
  end
  return ""
end

%    \end{macrocode}
%
%    Our MetaPost preambles
%    \begin{macrocode}
local mplibcodepreamble = [[
texscriptmode := 2;
def rawtextext (expr t) = runscript("luamplibtext{"&t&"}") enddef;
def mplibcolor (expr t) = runscript("luamplibcolor{"&t&"}") enddef;
def mplibdimen (expr t) = runscript("luamplibdimen{"&t&"}") enddef;
def VerbatimTeX (expr t) = runscript("luamplibverbtex{"&t&"}") enddef;
if known context_mlib:
  defaultfont := "cmtt10";
  let infont = normalinfont;
  let fontsize = normalfontsize;
  vardef thelabel@#(expr p,z) =
    if string p :
      thelabel@#(p infont defaultfont scaled defaultscale,z)
    else :
      p shifted (z + labeloffset*mfun_laboff@# -
        (mfun_labxf@#*lrcorner p + mfun_labyf@#*ulcorner p +
        (1-mfun_labxf@#-mfun_labyf@#)*llcorner p))
    fi
  enddef;
  def graphictext primary filename =
    if (readfrom filename = EOF):
      errmessage "Please prepare '"&filename&"' in advance with"&
      " 'pstoedit -ssp -dt -f mpost yourfile.ps "&filename&"'";
    fi
    closefrom filename;
    def data_mpy_file = filename enddef;
    mfun_do_graphic_text (filename)
  enddef;
else:
  vardef textext@# (text t) = rawtextext (t) enddef;
fi
def externalfigure primary filename =
  draw rawtextext("\includegraphics{"& filename &"}")
enddef;
def TEX = textext enddef;
]]
luamplib.mplibcodepreamble = mplibcodepreamble

local legacyverbatimtexpreamble = [[
def specialVerbatimTeX (text t) = runscript("luamplibprefig{"&t&"}") enddef;
def normalVerbatimTeX  (text t) = runscript("luamplibinfig{"&t&"}") enddef;
let VerbatimTeX = specialVerbatimTeX;
extra_beginfig := extra_beginfig & " let VerbatimTeX = normalVerbatimTeX;"&
  "runscript(" &ditto& "luamplib.in_the_fig=true" &ditto& ");";
extra_endfig := extra_endfig & " let VerbatimTeX = specialVerbatimTeX;"&
  "runscript(" &ditto&
  "if luamplib.in_the_fig then luamplib.figid=luamplib.figid+1 end "&
  "luamplib.in_the_fig=false" &ditto& ");";
]]
luamplib.legacyverbatimtexpreamble = legacyverbatimtexpreamble

local textextlabelpreamble = [[
primarydef s infont f = rawtextext(s) enddef;
def fontsize expr f =
  begingroup
  save size; numeric size;
  size := mplibdimen("1em");
  if size = 0: 10pt else: size fi
  endgroup
enddef;
]]
luamplib.textextlabelpreamble = textextlabelpreamble

%    \end{macrocode}
%
%    When \cs{mplibverbatim} is enabled, do not expand |mplibcode| data.
%    \begin{macrocode}
luamplib.verbatiminput = false

%    \end{macrocode}
%
%    Do not expand |btex ... etex|, |verbatimtex ... etex|, and
%    string expressions.
%    \begin{macrocode}
local function protect_expansion (str)
  if str then
    str = str:gsub("\\","!!!Control!!!")
             :gsub("%%","!!!Comment!!!")
             :gsub("#", "!!!HashSign!!!")
             :gsub("{", "!!!LBrace!!!")
             :gsub("}", "!!!RBrace!!!")
    return format("\\unexpanded{%s}",str)
  end
end

local function unprotect_expansion (str)
  if str then
    return str:gsub("!!!Control!!!", "\\")
              :gsub("!!!Comment!!!", "%%")
              :gsub("!!!HashSign!!!","#")
              :gsub("!!!LBrace!!!",  "{")
              :gsub("!!!RBrace!!!",  "}")
  end
end

luamplib.everymplib    = { [""] = "" }
luamplib.everyendmplib = { [""] = "" }

local function process_mplibcode (data, instancename)
%    \end{macrocode}
%
%    This is needed for legacy behavior regarding |verbatimtex|
%    \begin{macrocode}
  legacy_mplibcode_reset()

  local everymplib    = luamplib.everymplib[instancename] or
                        luamplib.everymplib[""]
  local everyendmplib = luamplib.everyendmplib[instancename] or
                        luamplib.everyendmplib[""]
  data = format("\n%s\n%s\n%s\n",everymplib, data, everyendmplib)
  data = data:gsub("\r","\n")

  data = data:gsub("\\mpcolor%s+(.-%b{})","mplibcolor(\"%1\")")
  data = data:gsub("\\mpdim%s+(%b{})", "mplibdimen(\"%1\")")
  data = data:gsub("\\mpdim%s+(\\%a+)","mplibdimen(\"%1\")")

  data = data:gsub(btex_etex, function(str)
    return format("btex %s etex ", -- space
      luamplib.verbatiminput and str or protect_expansion(str))
  end)
  data = data:gsub(verbatimtex_etex, function(str)
    return format("verbatimtex %s etex;", -- semicolon
      luamplib.verbatiminput and str or protect_expansion(str))
  end)

%    \end{macrocode}
%
%    If not |mplibverbatim|, expand |mplibcode| data,
%    so that users can use \TeX\ codes in it.
%    It has turned out that no comment sign is allowed.
%    \begin{macrocode}
  if not luamplib.verbatiminput then
    data = data:gsub("\".-\"", protect_expansion)

    data = data:gsub("\\%%", "\0PerCent\0")
    data = data:gsub("%%.-\n","")
    data = data:gsub("%zPerCent%z", "\\%%")

    run_tex_code(format("\\mplibtmptoks\\expanded{{%s}}",data))
    data = texgettoks"mplibtmptoks"
%    \end{macrocode}
%
%    Next line to address issue \#55
%    \begin{macrocode}
    data = data:gsub("##", "#")
    data = data:gsub("\".-\"", unprotect_expansion)
    data = data:gsub(btex_etex, function(str)
      return format("btex %s etex", unprotect_expansion(str))
    end)
    data = data:gsub(verbatimtex_etex, function(str)
      return format("verbatimtex %s etex", unprotect_expansion(str))
    end)
  end

  process(data, instancename)
end
luamplib.process_mplibcode = process_mplibcode

%    \end{macrocode}
%
%    For parsing |prescript| materials.
%    \begin{macrocode}
local further_split_keys = {
  mplibtexboxid = true,
  sh_color_a    = true,
  sh_color_b    = true,
}

local function script2table(s)
  local t = {}
  for _,i in ipairs(s:explode("\13+")) do
    local k,v = i:match("(.-)=(.*)") -- v may contain = or empty.
    if k and v and k ~= "" then
      if further_split_keys[k] then
        t[k] = v:explode(":")
      else
        t[k] = v
      end
    end
  end
  return t
end

%    \end{macrocode}
%
%    Codes below for inserting PDF lieterals are mostly from ConTeXt general,
%    with small changes when needed.
%    \begin{macrocode}
local function getobjects(result,figure,f)
  return figure:objects()
end

local function convert(result, flusher)
  luamplib.flush(result, flusher)
  return true -- done
end
luamplib.convert = convert

local function pdf_startfigure(n,llx,lly,urx,ury)
  texsprint(format("\\mplibstarttoPDF{%f}{%f}{%f}{%f}",llx,lly,urx,ury))
end

local function pdf_stopfigure()
  texsprint("\\mplibstoptoPDF")
end

%    \end{macrocode}
%
%    |tex.tprint| with catcode regime -2, as sometimes |#| gets doubled
%    in the argument of pdfliteral.
%    \begin{macrocode}
local function pdf_literalcode(fmt,...) -- table
  textprint({"\\mplibtoPDF{"},{-2,format(fmt,...)},{"}"})
end

local function pdf_textfigure(font,size,text,width,height,depth)
  text = text:gsub(".",function(c)
    return format("\\hbox{\\char%i}",string.byte(c)) -- kerning happens in metapost
  end)
  texsprint(format("\\mplibtextext{%s}{%f}{%s}{%s}{%f}",font,size,text,0,-( 7200/ 7227)/65536*depth))
end

local bend_tolerance = 131/65536

local rx, sx, sy, ry, tx, ty, divider = 1, 0, 0, 1, 0, 0, 1

local function pen_characteristics(object)
  local t = mplib.pen_info(object)
  rx, ry, sx, sy, tx, ty = t.rx, t.ry, t.sx, t.sy, t.tx, t.ty
  divider = sx*sy - rx*ry
  return not (sx==1 and rx==0 and ry==0 and sy==1 and tx==0 and ty==0), t.width
end

local function concat(px, py) -- no tx, ty here
  return (sy*px-ry*py)/divider,(sx*py-rx*px)/divider
end

local function curved(ith,pth)
  local d = pth.left_x - ith.right_x
  if abs(ith.right_x - ith.x_coord - d) <= bend_tolerance and abs(pth.x_coord - pth.left_x - d) <= bend_tolerance then
    d = pth.left_y - ith.right_y
    if abs(ith.right_y - ith.y_coord - d) <= bend_tolerance and abs(pth.y_coord - pth.left_y - d) <= bend_tolerance then
      return false
    end
  end
  return true
end

local function flushnormalpath(path,open)
  local pth, ith
  for i=1,#path do
    pth = path[i]
    if not ith then
      pdf_literalcode("%f %f m",pth.x_coord,pth.y_coord)
    elseif curved(ith,pth) then
      pdf_literalcode("%f %f %f %f %f %f c",ith.right_x,ith.right_y,pth.left_x,pth.left_y,pth.x_coord,pth.y_coord)
    else
      pdf_literalcode("%f %f l",pth.x_coord,pth.y_coord)
    end
    ith = pth
  end
  if not open then
    local one = path[1]
    if curved(pth,one) then
      pdf_literalcode("%f %f %f %f %f %f c",pth.right_x,pth.right_y,one.left_x,one.left_y,one.x_coord,one.y_coord )
    else
      pdf_literalcode("%f %f l",one.x_coord,one.y_coord)
    end
  elseif #path == 1 then -- special case .. draw point
    local one = path[1]
    pdf_literalcode("%f %f l",one.x_coord,one.y_coord)
  end
end

local function flushconcatpath(path,open)
  pdf_literalcode("%f %f %f %f %f %f cm", sx, rx, ry, sy, tx ,ty)
  local pth, ith
  for i=1,#path do
    pth = path[i]
    if not ith then
      pdf_literalcode("%f %f m",concat(pth.x_coord,pth.y_coord))
    elseif curved(ith,pth) then
      local a, b = concat(ith.right_x,ith.right_y)
      local c, d = concat(pth.left_x,pth.left_y)
      pdf_literalcode("%f %f %f %f %f %f c",a,b,c,d,concat(pth.x_coord, pth.y_coord))
    else
      pdf_literalcode("%f %f l",concat(pth.x_coord, pth.y_coord))
    end
    ith = pth
  end
  if not open then
    local one = path[1]
    if curved(pth,one) then
      local a, b = concat(pth.right_x,pth.right_y)
      local c, d = concat(one.left_x,one.left_y)
      pdf_literalcode("%f %f %f %f %f %f c",a,b,c,d,concat(one.x_coord, one.y_coord))
    else
      pdf_literalcode("%f %f l",concat(one.x_coord,one.y_coord))
    end
  elseif #path == 1 then -- special case .. draw point
    local one = path[1]
    pdf_literalcode("%f %f l",concat(one.x_coord,one.y_coord))
  end
end

%    \end{macrocode}
%
%    |dvipdfmx| is supported, though nobody seems to use it.
%    \begin{macrocode}
local pdfoutput = tonumber(texget("outputmode")) or tonumber(texget("pdfoutput"))
local pdfmode = pdfoutput > 0

local function start_pdf_code()
  if pdfmode then
    pdf_literalcode("q")
  else
    texsprint("\\special{pdf:bcontent}") -- dvipdfmx
  end
end
local function stop_pdf_code()
  if pdfmode then
    pdf_literalcode("Q")
  else
    texsprint("\\special{pdf:econtent}") -- dvipdfmx
  end
end

%    \end{macrocode}
%
%    Now we process hboxes created from |btex ... etex| or
%    |textext(...)| or |TEX(...)|, all being the same internally.
%    \begin{macrocode}
local function put_tex_boxes (object,prescript)
  local box = prescript.mplibtexboxid
  local n,tw,th = box[1],tonumber(box[2]),tonumber(box[3])
  if n and tw and th then
    local op = object.path
    local first, second, fourth = op[1], op[2], op[4]
    local tx, ty = first.x_coord, first.y_coord
    local sx, rx, ry, sy = 1, 0, 0, 1
    if tw ~= 0 then
      sx = (second.x_coord - tx)/tw
      rx = (second.y_coord - ty)/tw
      if sx == 0 then sx = 0.00001 end
    end
    if th ~= 0 then
      sy = (fourth.y_coord - ty)/th
      ry = (fourth.x_coord - tx)/th
      if sy == 0 then sy = 0.00001 end
    end
    start_pdf_code()
    pdf_literalcode("%f %f %f %f %f %f cm",sx,rx,ry,sy,tx,ty)
    texsprint(format("\\mplibputtextbox{%i}",n))
    stop_pdf_code()
  end
end

%    \end{macrocode}
%
%    Colors and Transparency
%    \begin{macrocode}
local pdf_objs = {}
local token, getpageres, setpageres = newtoken or token
local pgf = { bye = "pgfutil@everybye", extgs = "pgf@sys@addpdfresource@extgs@plain" }

if pdfmode then -- respect luaotfload-colors
  getpageres = pdf.getpageresources or function() return pdf.pageresources end
  setpageres = pdf.setpageresources or function(s) pdf.pageresources = s end
else
  texsprint("\\special{pdf:obj @MPlibTr<<>>}",
            "\\special{pdf:obj @MPlibSh<<>>}")
end

local function update_pdfobjs (os)
  local on = pdf_objs[os]
  if on then
    return on,false
  end
  if pdfmode then
    on = pdf.immediateobj(os)
  else
    on = pdf_objs.cnt or 0
    pdf_objs.cnt = on + 1
  end
  pdf_objs[os] = on
  return on,true
end

local transparancy_modes = { [0] = "Normal",
  "Normal",       "Multiply",     "Screen",       "Overlay",
  "SoftLight",    "HardLight",    "ColorDodge",   "ColorBurn",
  "Darken",       "Lighten",      "Difference",   "Exclusion",
  "Hue",          "Saturation",   "Color",        "Luminosity",
  "Compatible",
}

local function update_tr_res(res,mode,opaq)
  local os = format("<</BM /%s/ca %.3f/CA %.3f/AIS false>>",mode,opaq,opaq)
  local on, new = update_pdfobjs(os)
  if new then
    if pdfmode then
      res = format("%s/MPlibTr%i %i 0 R",res,on,on)
    else
      if pgf.loaded then
        texsprint(format("\\csname %s\\endcsname{/MPlibTr%i%s}", pgf.extgs, on, os))
      else
        texsprint(format("\\special{pdf:put @MPlibTr<</MPlibTr%i%s>>}",on,os))
      end
    end
  end
  return res,on
end

local function tr_pdf_pageresources(mode,opaq)
  if token and pgf.bye and not pgf.loaded then
    pgf.loaded = token.create(pgf.bye).cmdname == "assign_toks"
    pgf.bye    = pgf.loaded and pgf.bye
  end
  local res, on_on, off_on = "", nil, nil
  res, off_on = update_tr_res(res, "Normal", 1)
  res, on_on  = update_tr_res(res, mode, opaq)
  if pdfmode then
    if res ~= "" then
      if pgf.loaded then
        texsprint(format("\\csname %s\\endcsname{%s}", pgf.extgs, res))
      else
        local tpr, n = getpageres() or "", 0
        tpr, n = tpr:gsub("/ExtGState<<", "%1"..res)
        if n == 0 then
          tpr = format("%s/ExtGState<<%s>>", tpr, res)
        end
        setpageres(tpr)
      end
    end
  else
    if not pgf.loaded then
      texsprint(format("\\special{pdf:put @resources<</ExtGState @MPlibTr>>}"))
    end
  end
  return on_on, off_on
end

%    \end{macrocode}
%
%    Shading with |metafun| format. (maybe legacy way)
%    \begin{macrocode}
local shading_res

local function shading_initialize ()
  shading_res = {}
  if pdfmode and luatexbase.callbacktypes.finish_pdffile then -- ltluatex
    local shading_obj = pdf.reserveobj()
    setpageres(format("%s/Shading %i 0 R",getpageres() or "",shading_obj))
    luatexbase.add_to_callback("finish_pdffile", function()
      pdf.immediateobj(shading_obj,format("<<%s>>",tableconcat(shading_res)))
      end, "luamplib.finish_pdffile")
    pdf_objs.finishpdf = true
  end
end

local function sh_pdfpageresources(shtype,domain,colorspace,colora,colorb,coordinates)
  if not shading_res then shading_initialize() end
  local os = format("<</FunctionType 2/Domain [ %s ]/C0 [ %s ]/C1 [ %s ]/N 1>>",
                    domain, colora, colorb)
  local funcobj = pdfmode and format("%i 0 R",update_pdfobjs(os)) or os
  os = format("<</ShadingType %i/ColorSpace /%s/Function %s/Coords [ %s ]/Extend [ true true ]/AntiAlias true>>",
              shtype, colorspace, funcobj, coordinates)
  local on, new = update_pdfobjs(os)
  if pdfmode then
    if new then
      local res = format("/MPlibSh%i %i 0 R", on, on)
      if pdf_objs.finishpdf then
        shading_res[#shading_res+1] = res
      else
        local pageres = getpageres() or ""
        if not pageres:find("/Shading<<.*>>") then
          pageres = pageres.."/Shading<<>>"
        end
        pageres = pageres:gsub("/Shading<<","%1"..res)
        setpageres(pageres)
      end
    end
  else
    if new then
      texsprint(format("\\special{pdf:put @MPlibSh<</MPlibSh%i%s>>}",on,os))
    end
    texsprint(format("\\special{pdf:put @resources<</Shading @MPlibSh>>}"))
  end
  return on
end

local function color_normalize(ca,cb)
  if #cb == 1 then
    if #ca == 4 then
      cb[1], cb[2], cb[3], cb[4] = 0, 0, 0, 1-cb[1]
    else -- #ca = 3
      cb[1], cb[2], cb[3] = cb[1], cb[1], cb[1]
    end
  elseif #cb == 3 then -- #ca == 4
    cb[1], cb[2], cb[3], cb[4] = 1-cb[1], 1-cb[2], 1-cb[3], 0
  end
end

local prev_override_color

local function do_preobj_color(object,prescript)
%    \end{macrocode}
%
%    transparency
%    \begin{macrocode}
  local opaq = prescript and prescript.tr_transparency
  local tron_no, troff_no
  if opaq then
    local mode = prescript.tr_alternative or 1
    mode = transparancy_modes[tonumber(mode)]
    tron_no, troff_no = tr_pdf_pageresources(mode,opaq)
    pdf_literalcode("/MPlibTr%i gs",tron_no)
  end
%    \end{macrocode}
%
%    color
%    \begin{macrocode}
  local override = prescript and prescript.MPlibOverrideColor
  if override then
    if pdfmode then
      pdf_literalcode(override)
      override = nil
    else
      texsprint(format("\\special{color push %s}",override))
      prev_override_color = override
    end
  else
    local cs = object.color
    if cs and #cs > 0 then
      pdf_literalcode(luamplib.colorconverter(cs))
      prev_override_color = nil
    elseif not pdfmode then
      override = prev_override_color
      if override then
        texsprint(format("\\special{color push %s}",override))
      end
    end
  end
%    \end{macrocode}
%
%    shading
%    \begin{macrocode}
  local sh_type = prescript and prescript.sh_type
  if sh_type then
    local domain  = prescript.sh_domain
    local centera = prescript.sh_center_a:explode()
    local centerb = prescript.sh_center_b:explode()
    for _,t in pairs({centera,centerb}) do
      for i,v in ipairs(t) do
        t[i] = format("%f",v)
      end
    end
    centera = tableconcat(centera," ")
    centerb = tableconcat(centerb," ")
    local colora  = prescript.sh_color_a or {0};
    local colorb  = prescript.sh_color_b or {1};
    for _,t in pairs({colora,colorb}) do
      for i,v in ipairs(t) do
        t[i] = format("%.3f",v)
      end
    end
    if #colora > #colorb then
      color_normalize(colora,colorb)
    elseif #colorb > #colora then
      color_normalize(colorb,colora)
    end
    local colorspace
    if     #colorb == 1 then colorspace = "DeviceGray"
    elseif #colorb == 3 then colorspace = "DeviceRGB"
    elseif #colorb == 4 then colorspace = "DeviceCMYK"
    else   return troff_no,override
    end
    colora = tableconcat(colora, " ")
    colorb = tableconcat(colorb, " ")
    local shade_no
    if sh_type == "linear" then
      local coordinates = tableconcat({centera,centerb}," ")
      shade_no = sh_pdfpageresources(2,domain,colorspace,colora,colorb,coordinates)
    elseif sh_type == "circular" then
      local radiusa = format("%f",prescript.sh_radius_a)
      local radiusb = format("%f",prescript.sh_radius_b)
      local coordinates = tableconcat({centera,radiusa,centerb,radiusb}," ")
      shade_no = sh_pdfpageresources(3,domain,colorspace,colora,colorb,coordinates)
    end
    pdf_literalcode("q /Pattern cs")
    return troff_no,override,shade_no
  end
  return troff_no,override
end

local function do_postobj_color(tr,over,sh)
  if sh then
    pdf_literalcode("W n /MPlibSh%s sh Q",sh)
  end
  if over then
    texsprint("\\special{color pop}")
  end
  if tr then
    pdf_literalcode("/MPlibTr%i gs",tr)
  end
end

%    \end{macrocode}
%
%    Finally, flush figures by inserting PDF literals.
%    \begin{macrocode}
local function flush(result,flusher)
  if result then
    local figures = result.fig
    if figures then
      for f=1, #figures do
        info("flushing figure %s",f)
        local figure = figures[f]
        local objects = getobjects(result,figure,f)
        local fignum = tonumber(figure:filename():match("([%d]+)$") or figure:charcode() or 0)
        local miterlimit, linecap, linejoin, dashed = -1, -1, -1, false
        local bbox = figure:boundingbox()
        local llx, lly, urx, ury = bbox[1], bbox[2], bbox[3], bbox[4] -- faster than unpack
        if urx < llx then
%    \end{macrocode}
%
%    luamplib silently ignores this invalid figure for those
%    that do not contain |beginfig ... endfig|. (issue \#70)
%    Original code of ConTeXt general was:
%    \begin{verbatim}
%    -- invalid
%    pdf_startfigure(fignum,0,0,0,0)
%    pdf_stopfigure()
%    \end{verbatim}
%    \begin{macrocode}
        else
%    \end{macrocode}
%
%    For legacy behavior. Insert `pre-fig' \TeX\ code here, and
%    prepare a table for `in-fig' codes.
%    \begin{macrocode}
          if tex_code_pre_mplib[f] then
            texsprint(tex_code_pre_mplib[f])
          end
          local TeX_code_bot = {}
          pdf_startfigure(fignum,llx,lly,urx,ury)
          start_pdf_code()
          if objects then
            local savedpath = nil
            local savedhtap = nil
            for o=1,#objects do
              local object        = objects[o]
              local objecttype    = object.type
%    \end{macrocode}
%
%    The following 5 lines are part of |btex...etex| patch.
%    Again, colors are processed at this stage.
%    \begin{macrocode}
              local prescript     = object.prescript
              prescript = prescript and script2table(prescript) -- prescript is now a table
              local tr_opaq,cr_over,shade_no = do_preobj_color(object,prescript)
              if prescript and prescript.mplibtexboxid then
                put_tex_boxes(object,prescript)
              elseif objecttype == "start_bounds" or objecttype == "stop_bounds" then --skip
              elseif objecttype == "start_clip" then
                local evenodd = not object.istext and object.postscript == "evenodd"
                start_pdf_code()
                flushnormalpath(object.path,false)
                pdf_literalcode(evenodd and "W* n" or "W n")
              elseif objecttype == "stop_clip" then
                stop_pdf_code()
                miterlimit, linecap, linejoin, dashed = -1, -1, -1, false
              elseif objecttype == "special" then
%    \end{macrocode}
%
%    Collect \TeX\ codes that will be executed after flushing.
%    Legacy behavior.
%    \begin{macrocode}
                if prescript and prescript.postmplibverbtex then
                  TeX_code_bot[#TeX_code_bot+1] = prescript.postmplibverbtex
                end
              elseif objecttype == "text" then
                local ot = object.transform -- 3,4,5,6,1,2
                start_pdf_code()
                pdf_literalcode("%f %f %f %f %f %f cm",ot[3],ot[4],ot[5],ot[6],ot[1],ot[2])
                pdf_textfigure(object.font,object.dsize,object.text,object.width,object.height,object.depth)
                stop_pdf_code()
              else
                local evenodd, collect, both = false, false, false
                local postscript = object.postscript
                if not object.istext then
                  if postscript == "evenodd" then
                    evenodd = true
                  elseif postscript == "collect" then
                    collect = true
                  elseif postscript == "both" then
                    both = true
                  elseif postscript == "eoboth" then
                    evenodd = true
                    both    = true
                  end
                end
                if collect then
                  if not savedpath then
                    savedpath = { object.path or false }
                    savedhtap = { object.htap or false }
                  else
                    savedpath[#savedpath+1] = object.path or false
                    savedhtap[#savedhtap+1] = object.htap or false
                  end
                else
                  local ml = object.miterlimit
                  if ml and ml ~= miterlimit then
                    miterlimit = ml
                    pdf_literalcode("%f M",ml)
                  end
                  local lj = object.linejoin
                  if lj and lj ~= linejoin then
                    linejoin = lj
                    pdf_literalcode("%i j",lj)
                  end
                  local lc = object.linecap
                  if lc and lc ~= linecap then
                    linecap = lc
                    pdf_literalcode("%i J",lc)
                  end
                  local dl = object.dash
                  if dl then
                    local d = format("[%s] %f d",tableconcat(dl.dashes or {}," "),dl.offset)
                    if d ~= dashed then
                      dashed = d
                      pdf_literalcode(dashed)
                    end
                  elseif dashed then
                    pdf_literalcode("[] 0 d")
                    dashed = false
                  end
                  local path = object.path
                  local transformed, penwidth = false, 1
                  local open = path and path[1].left_type and path[#path].right_type
                  local pen = object.pen
                  if pen then
                    if pen.type == 'elliptical' then
                      transformed, penwidth = pen_characteristics(object) -- boolean, value
                      pdf_literalcode("%f w",penwidth)
                      if objecttype == 'fill' then
                        objecttype = 'both'
                      end
                    else -- calculated by mplib itself
                      objecttype = 'fill'
                    end
                  end
                  if transformed then
                    start_pdf_code()
                  end
                  if path then
                    if savedpath then
                      for i=1,#savedpath do
                        local path = savedpath[i]
                        if transformed then
                          flushconcatpath(path,open)
                        else
                          flushnormalpath(path,open)
                        end
                      end
                      savedpath = nil
                    end
                    if transformed then
                      flushconcatpath(path,open)
                    else
                      flushnormalpath(path,open)
                    end
%    \end{macrocode}
%
%    Change from ConTeXt general: there was color stuffs.
%    \begin{macrocode}
                    if not shade_no then -- conflict with shading
                      if objecttype == "fill" then
                        pdf_literalcode(evenodd and "h f*" or "h f")
                      elseif objecttype == "outline" then
                        if both then
                          pdf_literalcode(evenodd and "h B*" or "h B")
                        else
                          pdf_literalcode(open and "S" or "h S")
                        end
                      elseif objecttype == "both" then
                        pdf_literalcode(evenodd and "h B*" or "h B")
                      end
                    end
                  end
                  if transformed then
                    stop_pdf_code()
                  end
                  local path = object.htap
                  if path then
                    if transformed then
                      start_pdf_code()
                    end
                    if savedhtap then
                      for i=1,#savedhtap do
                        local path = savedhtap[i]
                        if transformed then
                          flushconcatpath(path,open)
                        else
                          flushnormalpath(path,open)
                        end
                      end
                      savedhtap = nil
                      evenodd   = true
                    end
                    if transformed then
                      flushconcatpath(path,open)
                    else
                      flushnormalpath(path,open)
                    end
                    if objecttype == "fill" then
                      pdf_literalcode(evenodd and "h f*" or "h f")
                    elseif objecttype == "outline" then
                      pdf_literalcode(open and "S" or "h S")
                    elseif objecttype == "both" then
                      pdf_literalcode(evenodd and "h B*" or "h B")
                    end
                    if transformed then
                      stop_pdf_code()
                    end
                  end
                end
              end
%    \end{macrocode}
%
%    Added to ConTeXt general: color stuff.
%    And execute legacy |verbatimtex| code.
%    \begin{macrocode}
              do_postobj_color(tr_opaq,cr_over,shade_no)
            end
          end
          stop_pdf_code()
          pdf_stopfigure()
          if #TeX_code_bot > 0 then texsprint(TeX_code_bot) end
        end
      end
    end
  end
end
luamplib.flush = flush

local function colorconverter(cr)
  local n = #cr
  if n == 4 then
    local c, m, y, k = cr[1], cr[2], cr[3], cr[4]
    return format("%.3f %.3f %.3f %.3f k %.3f %.3f %.3f %.3f K",c,m,y,k,c,m,y,k), "0 g 0 G"
  elseif n == 3 then
    local r, g, b = cr[1], cr[2], cr[3]
    return format("%.3f %.3f %.3f rg %.3f %.3f %.3f RG",r,g,b,r,g,b), "0 g 0 G"
  else
    local s = cr[1]
    return format("%.3f g %.3f G",s,s), "0 g 0 G"
  end
end
luamplib.colorconverter = colorconverter
%    \end{macrocode}
%
% \iffalse
%</lua>
% \fi
%
%    \subsection{\texorpdfstring{\TeX}{TeX} package}
%
%
% \iffalse
%<*package>
% \fi
%
%    First we need to load some packages.
%
%    \begin{macrocode}
\bgroup\expandafter\expandafter\expandafter\egroup
\expandafter\ifx\csname selectfont\endcsname\relax
  \input ltluatex
\else
  \NeedsTeXFormat{LaTeX2e}
  \ProvidesPackage{luamplib}
    [2024/01/25 v2.25.3 mplib package for LuaTeX]
  \ifx\newluafunction\@undefined
  \input ltluatex
  \fi
\fi
%    \end{macrocode}
%
%    Loading of lua code.
%    \begin{macrocode}
\directlua{require("luamplib")}
%    \end{macrocode}
%
%    Support older engine. Seems we don't need it, but no harm.
%    \begin{macrocode}
\ifx\pdfoutput\undefined
  \let\pdfoutput\outputmode
  \protected\def\pdfliteral{\pdfextension literal}
\fi
%    \end{macrocode}
%
%    Unfortuantely there are still packages out there that think it is a good
%    idea to manually set \cs{pdfoutput} which defeats the above branch that
%    defines \cs{pdfliteral}.  To cover that case we need an extra check.
%    \begin{macrocode}
\ifx\pdfliteral\undefined
  \protected\def\pdfliteral{\pdfextension literal}
\fi
%    \end{macrocode}
%
%    Set the format for metapost.
%    \begin{macrocode}
\def\mplibsetformat#1{\directlua{luamplib.setformat("#1")}}
%    \end{macrocode}
%
%    luamplib works in both PDF and DVI mode,
%    but only DVIPDFMx is supported currently among a number of DVI tools.
%    So we output a info.
%    \begin{macrocode}
\ifnum\pdfoutput>0
  \let\mplibtoPDF\pdfliteral
\else
  \def\mplibtoPDF#1{\special{pdf:literal direct #1}}
  \ifcsname PackageInfo\endcsname
    \PackageInfo{luamplib}{take dvipdfmx path, no support for other dvi tools currently.}
  \else
    \write128{}
    \write128{luamplib Info: take dvipdfmx path, no support for other dvi tools currently.}
    \write128{}
  \fi
\fi
%    \end{macrocode}
%
%    Make |mplibcode| typesetted always in horizontal mode.
%    \begin{macrocode}
\def\mplibforcehmode{\let\prependtomplibbox\leavevmode}
\def\mplibnoforcehmode{\let\prependtomplibbox\relax}
\mplibnoforcehmode
%    \end{macrocode}
%
%    Catcode. We want to allow comment sign in |mplibcode|.
%    \begin{macrocode}
\def\mplibsetupcatcodes{%
  %catcode`\{=12 %catcode`\}=12
  \catcode`\#=12 \catcode`\^=12 \catcode`\~=12 \catcode`\_=12
  \catcode`\&=12 \catcode`\$=12 \catcode`\%=12 \catcode`\^^M=12
}
%    \end{macrocode}
%
%    Make |btex...etex| box zero-metric.
%    \begin{macrocode}
\def\mplibputtextbox#1{\vbox to 0pt{\vss\hbox to 0pt{\raise\dp#1\copy#1\hss}}}
%    \end{macrocode}
%
%    The Plain-specific stuff.
%    \begin{macrocode}
\unless\ifcsname ver@luamplib.sty\endcsname
\def\mplibcode{%
  \begingroup
  \begingroup
  \mplibsetupcatcodes
  \mplibdocode
}
\long\def\mplibdocode#1\endmplibcode{%
  \endgroup
  \directlua{luamplib.process_mplibcode([===[\unexpanded{#1}]===],"")}%
  \endgroup
}
\else
%    \end{macrocode}
%
%    The \LaTeX-specific part: a new environment.
%    \begin{macrocode}
\newenvironment{mplibcode}[1][]{%
  \global\def\currentmpinstancename{#1}%
  \mplibtmptoks{}\ltxdomplibcode
}{}
\def\ltxdomplibcode{%
  \begingroup
  \mplibsetupcatcodes
  \ltxdomplibcodeindeed
}
\def\mplib@mplibcode{mplibcode}
\long\def\ltxdomplibcodeindeed#1\end#2{%
  \endgroup
  \mplibtmptoks\expandafter{\the\mplibtmptoks#1}%
  \def\mplibtemp@a{#2}%
  \ifx\mplib@mplibcode\mplibtemp@a
    \directlua{luamplib.process_mplibcode([===[\the\mplibtmptoks]===],"\currentmpinstancename")}%
    \end{mplibcode}%
  \else
    \mplibtmptoks\expandafter{\the\mplibtmptoks\end{#2}}%
    \expandafter\ltxdomplibcode
  \fi
}
\fi
%    \end{macrocode}
%
%    User settings.
%    \begin{macrocode}
\def\mplibshowlog#1{\directlua{
    local s = string.lower("#1")
    if s == "enable" or s == "true" or s == "yes" then
      luamplib.showlog = true
    else
      luamplib.showlog = false
    end
}}
\def\mpliblegacybehavior#1{\directlua{
    local s = string.lower("#1")
    if s == "enable" or s == "true" or s == "yes" then
      luamplib.legacy_verbatimtex = true
    else
      luamplib.legacy_verbatimtex = false
    end
}}
\def\mplibverbatim#1{\directlua{
    local s = string.lower("#1")
    if s == "enable" or s == "true" or s == "yes" then
      luamplib.verbatiminput = true
    else
      luamplib.verbatiminput = false
    end
}}
\newtoks\mplibtmptoks
%    \end{macrocode}
%
%    \cs{everymplib} \& \cs{everyendmplib}: macros resetting
%    |luamplib.every(end)mplib| tables
%
%    \begin{macrocode}
\protected\def\everymplib{%
  \begingroup
  \mplibsetupcatcodes
  \mplibdoeverymplib
}
\protected\def\everyendmplib{%
  \begingroup
  \mplibsetupcatcodes
  \mplibdoeveryendmplib
}
\ifcsname ver@luamplib.sty\endcsname
  \newcommand\mplibdoeverymplib[2][]{%
    \endgroup
    \directlua{
      luamplib.everymplib["#1"] = [===[\unexpanded{#2}]===]
    }%
  }
  \newcommand\mplibdoeveryendmplib[2][]{%
    \endgroup
    \directlua{
      luamplib.everyendmplib["#1"] = [===[\unexpanded{#2}]===]
    }%
  }
\else
  \long\def\mplibdoeverymplib#1{%
    \endgroup
    \directlua{
      luamplib.everymplib[""] = [===[\unexpanded{#1}]===]
    }%
  }
  \long\def\mplibdoeveryendmplib#1{%
    \endgroup
    \directlua{
      luamplib.everyendmplib[""] = [===[\unexpanded{#1}]===]
    }%
  }
\fi
%    \end{macrocode}
%
%    Allow \TeX\ dimen/color macros. Now |runscript| does the job,
%    so the following lines are not needed for most cases.
%    But the macros will be expanded when they are used in another macro.
%    \begin{macrocode}
\def\mpdim#1{ mplibdimen("#1") }
\def\mpcolor#1#{\domplibcolor{#1}}
\def\domplibcolor#1#2{ mplibcolor("#1{#2}") }
%    \end{macrocode}
%
%    MPLib's number system. Now |binary| has gone away.
%    \begin{macrocode}
\def\mplibnumbersystem#1{\directlua{
  local t = "#1"
  if t == "binary" then t = "decimal" end
  luamplib.numbersystem = t
}}
%    \end{macrocode}
%
%    Settings for |.mp| cache files.
%    \begin{macrocode}
\def\mplibmakenocache#1{\mplibdomakenocache #1,*,}
\def\mplibdomakenocache#1,{%
  \ifx\empty#1\empty
    \expandafter\mplibdomakenocache
  \else
    \ifx*#1\else
      \directlua{luamplib.noneedtoreplace["#1.mp"]=true}%
      \expandafter\expandafter\expandafter\mplibdomakenocache
    \fi
  \fi
}
\def\mplibcancelnocache#1{\mplibdocancelnocache #1,*,}
\def\mplibdocancelnocache#1,{%
  \ifx\empty#1\empty
    \expandafter\mplibdocancelnocache
  \else
    \ifx*#1\else
      \directlua{luamplib.noneedtoreplace["#1.mp"]=false}%
      \expandafter\expandafter\expandafter\mplibdocancelnocache
    \fi
  \fi
}
\def\mplibcachedir#1{\directlua{luamplib.getcachedir("\unexpanded{#1}")}}
%    \end{macrocode}
%
%    More user settings.
%    \begin{macrocode}
\def\mplibtextextlabel#1{\directlua{
    local s = string.lower("#1")
    if s == "enable" or s == "true" or s == "yes" then
      luamplib.textextlabel = true
    else
      luamplib.textextlabel = false
    end
}}
\def\mplibcodeinherit#1{\directlua{
    local s = string.lower("#1")
    if s == "enable" or s == "true" or s == "yes" then
      luamplib.codeinherit = true
    else
      luamplib.codeinherit = false
    end
}}
\def\mplibglobaltextext#1{\directlua{
    local s = string.lower("#1")
    if s == "enable" or s == "true" or s == "yes" then
      luamplib.globaltextext = true
    else
      luamplib.globaltextext = false
    end
}}
%    \end{macrocode}
%
%    The followings are from ConTeXt general, mostly.
%    %    We use a dedicated scratchbox.
%    \begin{macrocode}
\ifx\mplibscratchbox\undefined \newbox\mplibscratchbox \fi
%    \end{macrocode}
%
%    We encapsulate the litterals.
%    \begin{macrocode}
\def\mplibstarttoPDF#1#2#3#4{%
  \prependtomplibbox
  \hbox\bgroup
  \xdef\MPllx{#1}\xdef\MPlly{#2}%
  \xdef\MPurx{#3}\xdef\MPury{#4}%
  \xdef\MPwidth{\the\dimexpr#3bp-#1bp\relax}%
  \xdef\MPheight{\the\dimexpr#4bp-#2bp\relax}%
  \parskip0pt%
  \leftskip0pt%
  \parindent0pt%
  \everypar{}%
  \setbox\mplibscratchbox\vbox\bgroup
  \noindent
}
\def\mplibstoptoPDF{%
  \par
  \egroup %
  \setbox\mplibscratchbox\hbox %
    {\hskip-\MPllx bp%
     \raise-\MPlly bp%
     \box\mplibscratchbox}%
  \setbox\mplibscratchbox\vbox to \MPheight
    {\vfill
     \hsize\MPwidth
     \wd\mplibscratchbox0pt%
     \ht\mplibscratchbox0pt%
     \dp\mplibscratchbox0pt%
     \box\mplibscratchbox}%
  \wd\mplibscratchbox\MPwidth
  \ht\mplibscratchbox\MPheight
  \box\mplibscratchbox
  \egroup
}
%    \end{macrocode}
%
%    Text items have a special handler.
%    \begin{macrocode}
\def\mplibtextext#1#2#3#4#5{%
  \begingroup
  \setbox\mplibscratchbox\hbox
    {\font\temp=#1 at #2bp%
     \temp
     #3}%
  \setbox\mplibscratchbox\hbox
    {\hskip#4 bp%
     \raise#5 bp%
     \box\mplibscratchbox}%
  \wd\mplibscratchbox0pt%
  \ht\mplibscratchbox0pt%
  \dp\mplibscratchbox0pt%
  \box\mplibscratchbox
  \endgroup
}
%    \end{macrocode}
%
%    Input |luamplib.cfg| when it exists.
%    \begin{macrocode}
\openin0=luamplib.cfg
\ifeof0 \else
  \closein0
  \input luamplib.cfg
\fi
%    \end{macrocode}
%
%    That's all folks!
%
% \iffalse
%</package>
% \fi
%
% \clearpage
% \section{The GNU GPL License v2}
%
% The GPL requires the complete license text to be distributed along
% with the code. I recommend the canonical source, instead:
% \url{http://www.gnu.org/licenses/old-licenses/gpl-2.0.html}.
% But if you insist on an included copy, here it is.
% You might want to zoom in.
%
% \newsavebox{\gpl}
% \begin{lrbox}{\gpl}
% \begin{minipage}{3\textwidth}
% \columnsep=3\columnsep
% \begin{multicols}{3}
% \begin{center}
% {\Large GNU GENERAL PUBLIC LICENSE\par}
% \bigskip
% {Version 2, June 1991}
% \end{center}
%
% \begin{center}
% {\parindent 0in
%
% Copyright \textcopyright\ 1989, 1991 Free Software Foundation, Inc.
%
% \bigskip
%
% 51 Franklin Street, Fifth Floor, Boston, MA  02110-1301, USA
%
% \bigskip
%
% Everyone is permitted to copy and distribute verbatim copies
% of this license document, but changing it is not allowed.
% }
% \end{center}
%
% \begin{center}
% {\bf\large Preamble}
% \end{center}
%
%
% The licenses for most software are designed to take away your freedom to
% share and change it.  By contrast, the GNU General Public License is
% intended to guarantee your freedom to share and change free software---to
% make sure the software is free for all its users.  This General Public
% License applies to most of the Free Software Foundation's software and to
% any other program whose authors commit to using it.  (Some other Free
% Software Foundation software is covered by the GNU Library General Public
% License instead.)  You can apply it to your programs, too.
%
% When we speak of free software, we are referring to freedom, not price.
% Our General Public Licenses are designed to make sure that you have the
% freedom to distribute copies of free software (and charge for this service
% if you wish), that you receive source code or can get it if you want it,
% that you can change the software or use pieces of it in new free programs;
% and that you know you can do these things.
%
% To protect your rights, we need to make restrictions that forbid anyone to
% deny you these rights or to ask you to surrender the rights.  These
% restrictions translate to certain responsibilities for you if you
% distribute copies of the software, or if you modify it.
%
% For example, if you distribute copies of such a program, whether gratis or
% for a fee, you must give the recipients all the rights that you have.  You
% must make sure that they, too, receive or can get the source code.  And
% you must show them these terms so they know their rights.
%
% We protect your rights with two steps: (1) copyright the software, and (2)
% offer you this license which gives you legal permission to copy,
% distribute and/or modify the software.
%
% Also, for each author's protection and ours, we want to make certain that
% everyone understands that there is no warranty for this free software.  If
% the software is modified by someone else and passed on, we want its
% recipients to know that what they have is not the original, so that any
% problems introduced by others will not reflect on the original authors'
% reputations.
%
% Finally, any free program is threatened constantly by software patents.
% We wish to avoid the danger that redistributors of a free program will
% individually obtain patent licenses, in effect making the program
% proprietary.  To prevent this, we have made it clear that any patent must
% be licensed for everyone's free use or not licensed at all.
%
% The precise terms and conditions for copying, distribution and
% modification follow.
%
% \begin{center}
% {\Large \sc Terms and Conditions For Copying, Distribution and
%   Modification}
% \end{center}
%
% \begin{enumerate}
% \item
% This License applies to any program or other work which contains a notice
% placed by the copyright holder saying it may be distributed under the
% terms of this General Public License.  The ``Program'', below, refers to
% any such program or work, and a ``work based on the Program'' means either
% the Program or any derivative work under copyright law: that is to say, a
% work containing the Program or a portion of it, either verbatim or with
% modifications and/or translated into another language.  (Hereinafter,
% translation is included without limitation in the term ``modification''.)
% Each licensee is addressed as ``you''.
%
% Activities other than copying, distribution and modification are not
% covered by this License; they are outside its scope.  The act of
% running the Program is not restricted, and the output from the Program
% is covered only if its contents constitute a work based on the
% Program (independent of having been made by running the Program).
% Whether that is true depends on what the Program does.
%
% \item You may copy and distribute verbatim copies of the Program's source
%   code as you receive it, in any medium, provided that you conspicuously
%   and appropriately publish on each copy an appropriate copyright notice
%   and disclaimer of warranty; keep intact all the notices that refer to
%   this License and to the absence of any warranty; and give any other
%   recipients of the Program a copy of this License along with the Program.
%
% You may charge a fee for the physical act of transferring a copy, and you
% may at your option offer warranty protection in exchange for a fee.
%
% \item
% You may modify your copy or copies of the Program or any portion
% of it, thus forming a work based on the Program, and copy and
% distribute such modifications or work under the terms of Section 1
% above, provided that you also meet all of these conditions:
%
% \begin{enumerate}
%
% \item
% You must cause the modified files to carry prominent notices stating that
% you changed the files and the date of any change.
%
% \item
% You must cause any work that you distribute or publish, that in
% whole or in part contains or is derived from the Program or any
% part thereof, to be licensed as a whole at no charge to all third
% parties under the terms of this License.
%
% \item
% If the modified program normally reads commands interactively
% when run, you must cause it, when started running for such
% interactive use in the most ordinary way, to print or display an
% announcement including an appropriate copyright notice and a
% notice that there is no warranty (or else, saying that you provide
% a warranty) and that users may redistribute the program under
% these conditions, and telling the user how to view a copy of this
% License.  (Exception: if the Program itself is interactive but
% does not normally print such an announcement, your work based on
% the Program is not required to print an announcement.)
%
% \end{enumerate}
%
%
% These requirements apply to the modified work as a whole.  If
% identifiable sections of that work are not derived from the Program,
% and can be reasonably considered independent and separate works in
% themselves, then this License, and its terms, do not apply to those
% sections when you distribute them as separate works.  But when you
% distribute the same sections as part of a whole which is a work based
% on the Program, the distribution of the whole must be on the terms of
% this License, whose permissions for other licensees extend to the
% entire whole, and thus to each and every part regardless of who wrote it.
%
% Thus, it is not the intent of this section to claim rights or contest
% your rights to work written entirely by you; rather, the intent is to
% exercise the right to control the distribution of derivative or
% collective works based on the Program.
%
% In addition, mere aggregation of another work not based on the Program
% with the Program (or with a work based on the Program) on a volume of
% a storage or distribution medium does not bring the other work under
% the scope of this License.
%
% \item
% You may copy and distribute the Program (or a work based on it,
% under Section 2) in object code or executable form under the terms of
% Sections 1 and 2 above provided that you also do one of the following:
%
% \begin{enumerate}
%
% \item
%
% Accompany it with the complete corresponding machine-readable
% source code, which must be distributed under the terms of Sections
% 1 and 2 above on a medium customarily used for software interchange; or,
%
% \item
%
% Accompany it with a written offer, valid for at least three
% years, to give any third party, for a charge no more than your
% cost of physically performing source distribution, a complete
% machine-readable copy of the corresponding source code, to be
% distributed under the terms of Sections 1 and 2 above on a medium
% customarily used for software interchange; or,
%
% \item
%
% Accompany it with the information you received as to the offer
% to distribute corresponding source code.  (This alternative is
% allowed only for noncommercial distribution and only if you
% received the program in object code or executable form with such
% an offer, in accord with Subsection b above.)
%
% \end{enumerate}
%
%
% The source code for a work means the preferred form of the work for
% making modifications to it.  For an executable work, complete source
% code means all the source code for all modules it contains, plus any
% associated interface definition files, plus the scripts used to
% control compilation and installation of the executable.  However, as a
% special exception, the source code distributed need not include
% anything that is normally distributed (in either source or binary
% form) with the major components (compiler, kernel, and so on) of the
% operating system on which the executable runs, unless that component
% itself accompanies the executable.
%
% If distribution of executable or object code is made by offering
% access to copy from a designated place, then offering equivalent
% access to copy the source code from the same place counts as
% distribution of the source code, even though third parties are not
% compelled to copy the source along with the object code.
%
% \item
% You may not copy, modify, sublicense, or distribute the Program
% except as expressly provided under this License.  Any attempt
% otherwise to copy, modify, sublicense or distribute the Program is
% void, and will automatically terminate your rights under this License.
% However, parties who have received copies, or rights, from you under
% this License will not have their licenses terminated so long as such
% parties remain in full compliance.
%
% \item
% You are not required to accept this License, since you have not
% signed it.  However, nothing else grants you permission to modify or
% distribute the Program or its derivative works.  These actions are
% prohibited by law if you do not accept this License.  Therefore, by
% modifying or distributing the Program (or any work based on the
% Program), you indicate your acceptance of this License to do so, and
% all its terms and conditions for copying, distributing or modifying
% the Program or works based on it.
%
% \item
% Each time you redistribute the Program (or any work based on the
% Program), the recipient automatically receives a license from the
% original licensor to copy, distribute or modify the Program subject to
% these terms and conditions.  You may not impose any further
% restrictions on the recipients' exercise of the rights granted herein.
% You are not responsible for enforcing compliance by third parties to
% this License.
%
% \item
% If, as a consequence of a court judgment or allegation of patent
% infringement or for any other reason (not limited to patent issues),
% conditions are imposed on you (whether by court order, agreement or
% otherwise) that contradict the conditions of this License, they do not
% excuse you from the conditions of this License.  If you cannot
% distribute so as to satisfy simultaneously your obligations under this
% License and any other pertinent obligations, then as a consequence you
% may not distribute the Program at all.  For example, if a patent
% license would not permit royalty-free redistribution of the Program by
% all those who receive copies directly or indirectly through you, then
% the only way you could satisfy both it and this License would be to
% refrain entirely from distribution of the Program.
%
% If any portion of this section is held invalid or unenforceable under
% any particular circumstance, the balance of the section is intended to
% apply and the section as a whole is intended to apply in other
% circumstances.
%
% It is not the purpose of this section to induce you to infringe any
% patents or other property right claims or to contest validity of any
% such claims; this section has the sole purpose of protecting the
% integrity of the free software distribution system, which is
% implemented by public license practices.  Many people have made
% generous contributions to the wide range of software distributed
% through that system in reliance on consistent application of that
% system; it is up to the author/donor to decide if he or she is willing
% to distribute software through any other system and a licensee cannot
% impose that choice.
%
% This section is intended to make thoroughly clear what is believed to
% be a consequence of the rest of this License.
%
% \item
% If the distribution and/or use of the Program is restricted in
% certain countries either by patents or by copyrighted interfaces, the
% original copyright holder who places the Program under this License
% may add an explicit geographical distribution limitation excluding
% those countries, so that distribution is permitted only in or among
% countries not thus excluded.  In such case, this License incorporates
% the limitation as if written in the body of this License.
%
% \item
% The Free Software Foundation may publish revised and/or new versions
% of the General Public License from time to time.  Such new versions will
% be similar in spirit to the present version, but may differ in detail to
% address new problems or concerns.
%
% Each version is given a distinguishing version number.  If the Program
% specifies a version number of this License which applies to it and ``any
% later version'', you have the option of following the terms and conditions
% either of that version or of any later version published by the Free
% Software Foundation.  If the Program does not specify a version number of
% this License, you may choose any version ever published by the Free Software
% Foundation.
%
% \item
% If you wish to incorporate parts of the Program into other free
% programs whose distribution conditions are different, write to the author
% to ask for permission.  For software which is copyrighted by the Free
% Software Foundation, write to the Free Software Foundation; we sometimes
% make exceptions for this.  Our decision will be guided by the two goals
% of preserving the free status of all derivatives of our free software and
% of promoting the sharing and reuse of software generally.
%
% \begin{center}
% {\Large\sc
% No Warranty
% }
% \end{center}
%
% \item
% {\sc Because the program is licensed free of charge, there is no warranty
% for the program, to the extent permitted by applicable law.  Except when
% otherwise stated in writing the copyright holders and/or other parties
% provide the program ``as is'' without warranty of any kind, either expressed
% or implied, including, but not limited to, the implied warranties of
% merchantability and fitness for a particular purpose.  The entire risk as
% to the quality and performance of the program is with you.  Should the
% program prove defective, you assume the cost of all necessary servicing,
% repair or correction.}
%
% \item
% {\sc In no event unless required by applicable law or agreed to in writing
% will any copyright holder, or any other party who may modify and/or
% redistribute the program as permitted above, be liable to you for damages,
% including any general, special, incidental or consequential damages arising
% out of the use or inability to use the program (including but not limited
% to loss of data or data being rendered inaccurate or losses sustained by
% you or third parties or a failure of the program to operate with any other
% programs), even if such holder or other party has been advised of the
% possibility of such damages.}
%
% \end{enumerate}
%
%
% \begin{center}
% {\Large\sc End of Terms and Conditions}
% \end{center}
%
%
% \pagebreak[2]
%
% \section*{Appendix: How to Apply These Terms to Your New Programs}
%
% If you develop a new program, and you want it to be of the greatest
% possible use to the public, the best way to achieve this is to make it
% free software which everyone can redistribute and change under these
% terms.
%
%   To do so, attach the following notices to the program.  It is safest to
%   attach them to the start of each source file to most effectively convey
%   the exclusion of warranty; and each file should have at least the
%   ``copyright'' line and a pointer to where the full notice is found.
%
% \begin{quote}
% one line to give the program's name and a brief idea of what it does. \\
% Copyright (C) yyyy  name of author \\
%
% This program is free software; you can redistribute it and/or modify
% it under the terms of the GNU General Public License as published by
% the Free Software Foundation; either version 2 of the License, or
% (at your option) any later version.
%
% This program is distributed in the hope that it will be useful,
% but WITHOUT ANY WARRANTY; without even the implied warranty of
% MERCHANTABILITY or FITNESS FOR A PARTICULAR PURPOSE.  See the
% GNU General Public License for more details.
%
% You should have received a copy of the GNU General Public License
% along with this program; if not, write to the Free Software
% Foundation, Inc., 51 Franklin Street, Fifth Floor, Boston, MA  02110-1301, USA.
% \end{quote}
%
% Also add information on how to contact you by electronic and paper mail.
%
% If the program is interactive, make it output a short notice like this
% when it starts in an interactive mode:
%
% \begin{quote}
% Gnomovision version 69, Copyright (C) yyyy  name of author \\
% Gnomovision comes with ABSOLUTELY NO WARRANTY; for details type `show w'. \\
% This is free software, and you are welcome to redistribute it
% under certain conditions; type `show c' for details.
% \end{quote}
%
%
% The hypothetical commands {\tt show w} and {\tt show c} should show the
% appropriate parts of the General Public License.  Of course, the commands
% you use may be called something other than {\tt show w} and {\tt show c};
% they could even be mouse-clicks or menu items---whatever suits your
% program.
%
% You should also get your employer (if you work as a programmer) or your
% school, if any, to sign a ``copyright disclaimer'' for the program, if
% necessary.  Here is a sample; alter the names:
%
% \begin{quote}
% Yoyodyne, Inc., hereby disclaims all copyright interest in the program \\
% `Gnomovision' (which makes passes at compilers) written by James Hacker. \\
%
% signature of Ty Coon, 1 April 1989 \\
% Ty Coon, President of Vice
% \end{quote}
%
%
% This General Public License does not permit incorporating your program
% into proprietary programs.  If your program is a subroutine library, you
% may consider it more useful to permit linking proprietary applications
% with the library.  If this is what you want to do, use the GNU Library
% General Public License instead of this License.
%
% \end{multicols}
% \end{minipage}
% \end{lrbox}
%
% \begin{center}
% \scalebox{0.33}{\usebox{\gpl}}
% \end{center}
%
% \Finale
\endinput
